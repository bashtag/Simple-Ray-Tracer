\documentclass[11pt]{article}
\usepackage[a4paper,margin=1in]{geometry}
\usepackage{listings}
\usepackage{color}
\usepackage{hyperref}

\title{\textbf{RayTracer Project - Build \& Usage Documentation}}
\author{CSE 461 - Computer Graphics}
\date{}

\definecolor{codegray}{gray}{0.95}
\lstset{
  backgroundcolor=\color{codegray},
  basicstyle=\ttfamily\footnotesize,
  frame=single,
  breaklines=true,
  postbreak=\mbox{\textcolor{red}{$\hookrightarrow$}\space},
  showstringspaces=false
}

\begin{document}

\maketitle

\section*{Environment Requirements}
This project is designed to be built and run on \textbf{Ubuntu Linux}. Ensure you are working in a Unix-based environment with the following tools installed:

\begin{itemize}
    \item \texttt{g++} with C++17 support
    \item GNU Make
    \item A terminal or shell capable of running \texttt{make}, \texttt{bash}, and standard Linux utilities
    \item Makefile is provided
\end{itemize}

Tested on Ubuntu 22.04 LTS and compatible with most modern Debian-based distributions.

\section{Compiling the Project}
To compile the main raytracer and test runner binaries, use the following command:

\begin{lstlisting}
make
\end{lstlisting}

This will:
\begin{itemize}
    \item Compile the source files from \texttt{src/} and \texttt{lib/}
    \item Place object files in \texttt{build/obj/}
    \item Generate the final executable \texttt{raytracer} at:
    \begin{lstlisting}
    build/release/raytracer
    \end{lstlisting}
\end{itemize}

\section{Running the RayTracer}
You can run the raytracer manually by providing:
\begin{itemize}
    \item A scene XML file path
    \item An output PNG file path
    \item A render mode: \texttt{single} or \texttt{multi}
\end{itemize}

\textbf{Example command:}
\begin{lstlisting}
./build/release/raytracer ./assets/scenes/scene_low_tree.xml outputs/output_tree.png multi
\end{lstlisting}

This will render \texttt{scene\_low\_tree.xml} using multithreaded mode and save the output to \texttt{outputs/output\_tree.png}.

\section{Output Directory}
Rendered images are saved in the \texttt{outputs/} directory. This folder is automatically created if it does not exist.

\section{Running System Tests}
To build and run the test runner (which automatically renders all scenes in \texttt{assets/scenes/}):

\begin{lstlisting}
make tests
\end{lstlisting}

This will:
\begin{itemize}
    \item Build the test binary at \texttt{build/tests/test\_runner}
    \item Run the binary
    \item Render all scenes with multithreading
    \item Save output images as:
    \begin{lstlisting}
    outputs/output_scene_<scene_name>_<timestamp>.png
    \end{lstlisting}
\end{itemize}

\textbf{Example generated output:}
\begin{lstlisting}
outputs/output_scene_scene_low_tree_20250412_154015.png
\end{lstlisting}

\section{Cleaning Up}
To remove all compiled files and outputs:

\begin{lstlisting}
make clean
\end{lstlisting}

To remove only output images:

\begin{lstlisting}
make clean_outputs
\end{lstlisting}

To remove only object files:

\begin{lstlisting}
make clean_obj
\end{lstlisting}

\section*{Notes}
\begin{itemize}
    \item If \texttt{assets/scenes/} is missing, the test runner will exit with an error.
    \item If \texttt{assets/textures/} is missing, a warning will be shown, but the test will proceed.
    \item All timestamps are generated based on the local system time.
\end{itemize}

\end{document}
